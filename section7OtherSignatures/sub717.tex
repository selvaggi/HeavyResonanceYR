\subsubsection{\theoryC{HL- and HE-LHC sensitivity to 2HDMs with $U(1)_X$ Gauge Symmetries}}
\contributors{Daniel A. Camargo, Luigi Delle Rose, Stefano Moretti, Farinaldo S. Queiroz}
%{\bf Author(s): Daniel A. Camargo$^1$, Luigi Delle Rose$^2$, Stefano Moretti$^3$ and Farinaldo S. Queiroz$^{1}$}\\
%
%{\it \small $^1$ International Institute of Physics, Universidade Federal do Rio Grande do Norte, Campus Universitario, Lagoa Nova, Natal-RN 59078-970, Brazil}\\
%{\it \small $^2$ INFN, Sezione di Firenze, and Dipartimento di Fisica ed Astronomia, Universit\`a di Firenze,  Via G. Sansone 1, 50019 Sesto Fiorentino, Italy}\\
%{\it \small $^3$ School of Physics and Astronomy, University of Southampton, Highfield, Southampton SO17 1BJ, United Kingdom}\\
%{\small email: dacamargov@gmail.com, luigi.dellerose@fi.infn.it, S.Moretti@soton.ac.uk, farinaldo.queiroz@iip.ufrn.br }\\

%\begin{abstract}
%After well over  a decade of operation, the Large Hadron Collider (LHC) will probably  increase its luminosity and/or energy power in such a way that it will be able to test theories containing new fields at the TeV scale with high precision.  $U(1)_X$-gauged extensions of 2-Higgs Doublet Models (2HDMs) have addressed outstanding problems like Dark Matter (DM) existence and neutrino masses proposing hypothetical fields beyond the Standard Model (SM) living at the TeV scale, chiefly, in the form of a $Z'$ boson. In this contribution, we present up-to-date collider bounds on several such theoretical setups. Moreover, we show the potential sensitivity to such $Z^\prime$s of the proposed High Luminosity LHC (HL-LHC) and High Energy LHC (HE-LHC).
%\end{abstract}


%\keywords{HL-LHC, HE-LHC, 2HDMs, Gauge Symmetries.}

%\maketitle % this produces the title block
 
%\subsubsection{Introduction}
%\paragraph*{Introduction}
 
%The current run of LHC activity (Run 2) offers significantly increased performance and data volume with respect to the previous run (Run 1). 
%Hence, improved precision tests, thanks to an overall considerably better statistics, will come as a result of this making it possible 
%to assess the viability of several TeV scale theories of new physics 
%Beyond the Standard Model (BSM). So far, the SM has offered the best description of 
%Quantum Chromo-Dynamics (QCD) and Electro-Weak (EW) interactions and has successfully passed all precision tests~\cite{Novaes:1999yn}. Within it, masses are generated through the so-called Higgs mechanism via the presence of one 
%scalar doublet which gives rise to the $125$~GeV Higgs boson discovered some years ago at the LHC ~\cite{Chatrchyan:2012xdj, Aad:2012tfa}. 
%However, e
Extended Higgs sectors belonging to various BSM scenarios 
%are also able to encompass all the SM Higgs boson properties. In addition, though, they 
offer the possibility to 
solve some of the open problems of the SM. Such frameworks with extended Higgs sectors have recently come together with new $U(1)_X$ gauge symmetries, offering a natural 
solutions to the DM and the neutrino mass problems.
These scenarios \cite{Ko:2013zsa,Ko:2014uka,Berlin:2014cfa,Huang:2015wts,DelleRose:2017xil,Campos:2017dgc} predict
%For instance, assuming a 2HDM sector~, 
%alongside that of
%the ensuing additional Higgs bosons,   
a rich 
new phenomenology due to the presence of a massive $Z^\prime$ gauge 
boson arising after the $U(1)_X$ spontaneous symmetry breaking.
 
Our goal is to explore the potential of the HL- and HE-LHC to study such a scenario. To do so, we fist use the latest available dilepton data  
from the LHC~\cite{Aaboud:2017buh, Sirunyan:2018exx} to constrain the mass and couplings of such $Z^\prime$ gauge bosons and, consequently, the viable parameter space of the underlying model. Then we use this result to asses the capabilities of the HL- and HE-LHC to test the existence of such heavy neutral vector bosons. An extended version of this contribution can be found in \citeref{Camargo:2018klg}.
 
The relevant part of the Lagrangian of the model we consider is
\begin{eqnarray}
\mathcal{L_{\rm NC}} \supset & - \left(\frac{g_Z}{2} J_{NC}^\mu \cos \xi \right) Z_\mu- \left(\frac{g_Z}{2} J_{NC}^\mu \sin \xi \right) Z^\prime_\mu \\
& + \frac{1}{4} g_X \sin \xi \left[ \left( Q_{Xf} ^R + Q_{Xf} ^L \right) \bar{\psi} _f \gamma ^\mu \psi _f + \left( Q_{Xf} ^R - Q_{Xf} ^L \right) \bar{\psi} _f \gamma ^\mu \gamma _5 \psi _f \right] Z_\mu \\
& - \frac{1}{4} g_X \cos \xi \left[ \left( Q_{Xf} ^R + Q_{Xf} ^L \right) \bar{\psi} _f \gamma ^\mu \psi _f - \left( Q_{Xf} ^L - Q_{Xf} ^R \right) \bar{\psi} _f \gamma ^\mu \gamma _5 \psi _f \right] Z' _\mu,
\label{eq:LNC1}
\end{eqnarray}
%
where $\xi$ represents the $Z-Z^\prime$ mixing parameter, $g_X$ the gauge coupling of the new abelian symmetry while 
$Q^L_X$($Q^R_X$) are the left(right)-handed fermion charges under $U(1)_X$ 
defined according to \citeref{Camargo:2018klg}. This interaction Lagrangian
represents the key information for the collider phenomenology we are going to tackle, 
because it dictates $Z^\prime$ production rates at the LHC as well as its most prominent decays to be searched for.

%\subsubsection{LHC Bounds}
%\paragraph*{LHC Bounds}

Present LHC bounds are obtained here by simulating at $13$~TeV of \com energy the process
%
\begin{equation}
pp\rightarrow l^+l^- +X,
\end{equation}
%
where $l=e,\mu$, leading to dilepton signals, and $X$ represents the sourrounding hadronic activity. (The dijet signal case was studied in \citeref{Camargo:2018klg} and found to be less sensitive.) 
Since this channel is mediated by a heavy $Z^\prime$, alongside $\gamma$ and $Z$, a peak 
around the $Z^\prime$ mass would appear at large values of the invariant mass of the dilepton final state. 
We describe our results adopting both the Narrow Width Approximation (NWA) and also introducing Finite Width (FW) effects. 
In this respect, we have implemented in \feynrules~\cite{Alloul:2013bka} the models shown in \citeref{Camargo:2018klg}, where the corresponding $U(1)_X$ charge assignment is explicitly shown,
and we have simulated the partonic events with \MGvATNLO~\cite{Alwall:2011uj}. For hadronization and detector effects we used \pythia8~\cite{Sjostrand:2007gs} and \delphes~\cite{deFavereau:2013fsa}, respectively.

%\paragraph*{HL-LHC and HE-LHC Sensitivity}\label{par:sec_HL-HE-LHC}
Assuming, for the sake of definiteness, the NWA with $g_X=0.1, 0.2$ and $0.3$, we project the current experimental limits to the HL- and HE-LHC. To find their
physics sensitivity we adopt the strategy described in \citeref{Papucci:2014rja}. 
In short, what we do is to solve an equation for $M_{\rm new}$ 
(i.e., the new limit on $m_{Z'}$), knowing the current bound $M$, as follows:\rt{This procedure does not look correct to me. It's not signal, but background that should be rescaled.}
\begin{equation}
\frac{ N_{\rm signal\, events} (M_{\rm new}^2, E_{\rm new},\mathcal{L}_{\rm new})}
     { N_{\rm signal\, events} (M^2, 13\,{\rm TeV},36 {\rm fb}^{-1})}=1,
\end{equation}
with obvious meaning of the subscripts.

The results from this iteration are summarized in tables \ref{tab:LHCforcast}, \ref{tab:LHCforcast2} and \ref{tab:LHCforcast3}. 
The lower mass bounds found presently compared to the expected ones at the 
HL-LHC and/or HE-LHC clearly show how important is any LHC upgrade to test 
new physics models including a $Z'$. For some models and benchmark points, such as, e.g., $U(1)_A$ with $g_X =0.1$, 
the HE-LHC will potentially probe $Z^\prime$ masses up to $7$~TeV, while for others, such as, e.g.,  $U(1)_F$ with $g_X =0.3$, it will potentially exclude masses up to $12$~TeV. In short, both LHC upgrades can generally extend the current reach in $m_{Z'}$ by a factor 2 to 3. The results are summarized in the tables. 

\begin{table*}[!t]
\centering
%\begin{tabular*}{\columnwidth}{@{\extracolsep{\fill}}lllll@{}}
\small\begin{tabular}{ccccccccc}
\hline 
Model & $13$~TeV, 36 fb$^{-1}$ & $13$~TeV, 300 fb$^{-1}$ &  $13$~TeV, 3000 fb$^{-1}$ & $27$ TeV, 300 fb$^{-1}$  & $27$~TeV, 3000 fb$^{-1}$\\ \hline 
$U(1)_A$ & $2.2$~TeV & $3.07$~TeV & $4.09$~TeV & $5.02$~TeV & $7.03$~TeV \\
$U(1)_B$ & $2.2$~TeV & $3.07$~TeV & $4.09$~TeV & $5.02$~TeV & $7.03$~TeV \\
$U(1)_C$ & $1.6$~TeV & $2.37$~TeV & $3.34$~TeV & $3.73$~TeV & $5.54$~TeV \\
$U(1)_D$ & $3.5$~TeV & $4.45$~TeV & $5.46$~TeV & $7.76$~TeV & $9.89$~TeV \\
$U(1)_E$ & $2.3$~TeV & $3.18$~TeV & $4.21$~TeV & $5.24$~TeV & $7.27$~TeV \\
$U(1)_F$ & $3.6$~TeV & $4.55$~TeV & $5.56$~TeV & $7.97$~TeV & $10.09$~TeV \\
$U(1)_G$ & $1.1$~TeV & $1.73$~TeV & $2.60$~TeV & $2.62$~TeV & $4.16$~TeV \\
$U(1)_{B-L}$ & $2$~TeV & $2.84$~TeV & $3.85$~TeV & $4.60$~TeV  & $6.55$~TeV \\
\hline
\end{tabular}\normalsize
\caption{HL-LHC and HE-LHC projected sensitivities for all $U(1)_X$ models studied in this work
using dilepton data at $13$~TeV and $27$~TeV of \com energy and for $\mathcal{L}=36$, 300 and 3000 fb$^{-1}$. Here, $g_X =0.1$.}
\label{tab:LHCforcast}
\end{table*}


\begin{table*}[!t]
\centering
%\begin{tabular*}{\columnwidth}{@{\extracolsep{\fill}}lllll@{}}
\small\begin{tabular}{ccccccccc}
\hline 
Model & $13$~TeV, 36 fb$^{-1}$ & $13$~TeV, 300 fb$^{-1}$ &  $13$~TeV, 3000 fb$^{-1}$ & $27$ TeV, 300 fb$^{-1}$  & $27$~TeV, 3000 fb$^{-1}$\\ \hline 
$U(1)_A$ & $3.2$~TeV & $4.14$~TeV & $5.17$~TeV & $7.14$~TeV & $9.26$~TeV \\
$U(1)_B$ & $3.2$~TeV & $4.14$~TeV & $5.17$~TeV & $7.17$~TeV & $9.26$~TeV \\
$U(1)_C$ & $2.7$~TeV & $3.62$~TeV & $4.65$~TeV & $6.09$~TeV & $8.18$~TeV \\
$U(1)_D$ & $4.1$~TeV & $5$~TeV & $6$~TeV & $9$~TeV & $11.1$~TeV \\
$U(1)_E$ & $4.1$~TeV & $5$~TeV & $6$~TeV & $9$~TeV & $11.1$~TeV \\
$U(1)_F$ & $4.6$~TeV & $5.53$~TeV & $6.51$~TeV & $10.02$~TeV & $12.09$~TeV \\
$U(1)_G$ & $1.6$~TeV & $2.37$~TeV & $3.34$~TeV & $3.73$~TeV & $5.54$~TeV \\
$U(1)_{B-L}$ & $2.9$~TeV & $3.83$~TeV & $4.86$~TeV & $6.5$~TeV  & $8.62$~TeV \\
\hline
\end{tabular}\normalsize
\caption{HL-LHC and HE-LHC projected sensitivities for all $U(1)_X$ models studied in this work
using dilepton data at $13$~TeV and $27$~TeV of \com energy and for $\mathcal{L}=36$, 300 and 3000 fb$^{-1}$. Here, $g_X =0.2$.}
\label{tab:LHCforcast2}
\end{table*}

\begin{table*}[!t]
\centering
%\begin{tabular*}{\columnwidth}{@{\extracolsep{\fill}}lllll@{}}
\small\begin{tabular}{ccccccccc}
\hline 
Model & $13$~TeV, 36 fb$^{-1}$ & $13$~TeV, 300 fb$^{-1}$ &  $13$~TeV, 3000 fb$^{-1}$ & $27$ TeV, 300 fb$^{-1}$  & $27$~TeV, 3000 fb$^{-1}$\\ \hline 
$U(1)_A$ & $3.8$~TeV & $4.75$~TeV & $5.75$~TeV & $8.38$~TeV & $10.5$~TeV \\
$U(1)_B$ & $3.8$~TeV & $4.75$~TeV & $5.75$~TeV & $8.38$~TeV & $10.5$~TeV \\
$U(1)_C$ & $3.1$~TeV & $4$~TeV & $5$~TeV & $6.93$~TeV & $9.05$~TeV \\
$U(1)_D$ & $4.8$~TeV & $5.72$~TeV & $6.7$~TeV & $10.4$~TeV & $12.48$~TeV \\
$U(1)_E$ & $4.2$~TeV & $5.14$~TeV & $6.14$~TeV & $9.2$~TeV & $11.3$~TeV \\
$U(1)_F$ & $5.0$~TeV & $5.91$~TeV & $6.88$~TeV & $10.84$~TeV & $12.87$~TeV \\
$U(1)_G$ & $3.1$~TeV & $4$~TeV & $5.07$~TeV & $6.93$~TeV & $9.05$~TeV \\
$U(1)_{B-L}$ & $3.4$~TeV & $4.35$~TeV & $5.37$~TeV & $7.55$~TeV  & $9.68$~TeV \\
\hline
\end{tabular}\normalsize
\caption{HL-LHC and HE-LHC projected sensitivities for all $U(1)_X$ models studied in this work
using dilepton data at $13$~TeV and $27$~TeV of \com energy and for $\mathcal{L}=36$, 300 and 3000 fb$^{-1}$. Here, $g_X =0.3$.}
\label{tab:LHCforcast3}
\end{table*}

%\subsubsection{Conclusions}
%\paragraph*{Conclusions}
%
%
%We have presented the HL-LHC and HE-LHC sensitivities in the search for new heavy neutral vector bosons in several $U(1)_X$ scenarios with a 2HDM structure in their (pseudo)scalar sectors. 
%As a result, we found that the discuss upgrades of the LHC are capable of probing $Z^\prime$ masses well into the 10 TeV
%domain.

\subsubsubsection*{Acknowledgments}
\noindent
The authors thank Werner Rodejohann and Pyung-won Ko
for  discussions  and  comments.   DC  and  FSQ  acknowledge
financial support from MEC and UFRN. FSQ also acknowledges the ICTP-SAIFR FAPESP grant 2016/01343-7 for additional  financial  support.   SM  is  supported  in  part  by  the
NExT  Institute  and  acknowledges  partial  financial  support
from  the  STFC  Consolidated  Grant  ST/L000296/1  and  the
H2020-MSCA-RISE-2014 grant no.   645722 (NonMinimal-
Higgs).

