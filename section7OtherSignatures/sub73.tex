\subsubsection{\atlas{Prospects for $t\bar{t}$ resonance searches at HL-LHC}}
\contributors{A. Duncan, M. Wielers, G. Lee, M. Marjanovic et al. ATLAS}%\rt{This is not finished.}
%{\bf Author(s): A. Duncan, M. Wielers, G. Lee, M. Marjanovic et al. ATLAS}

HL-LHC prospects at for $Z^\prime$ bosons in the $t\bar{t}$ final state were presented in \citeref{ATL-PHYS-PUB-2017-002}
based on the event selection and systematic uncertainties from the Run~1 analysis with 20.3~fb$^{-1}$
collected at $\sqrt{s}=8$~TeV in \citeref{Aad:2015fna}.
These results have been updated using a more recent parameterisation of the $b$-tagging efficiencies and
misidentification rates as shown in \citeref{CERN-LHCC-2017-018} and are summarized below.

The analysis looks for a narrow width $Z^\prime$ boson
in a final state in which one of the $W$ bosons from the top quark decays to two jets and the other decays to a lepton (electron or muon) and a neutrino ($t\bar{t}\rightarrow WbWb \rightarrow \ell \nu b q q^{\prime}b$). Events are required to contain exactly one lepton, several jets and at least a moderate amount of missing transverse momentum must be present. Events are separated into boosted and resolved channels with most of the signal events falling in the former category. In the resolved channel the decay products of the hadronic top-quark decay are reconstructed as three separate jets and in total events must contain at least four jets. In the boosted channel, the hadronic top-quark decay products are highly boosted and end up in one broad large-radius jet. Events are selected if at least one large-radius jet and one jet (from the other top-quark decay) is present.
Subsequently $m_{t\bar{t}}$ is reconstructed based on the reconstruction of the $W$ bosons and $b$-jets in the event. Using $m_{t\bar{t}}$ as discriminant, upper limits are set on the signal cross section times branching ratio as a function of the $Z^\prime$ boson mass.

Using as benchmark a Topcolour-assisted Technicolour $Z^{\prime}_{\mathrm{TC2}}$ boson with a narrow width of 1.2\%, $Z^{\prime}_{\mathrm{TC2}}$ bosons can be excluded up to masses of $\simeq$ 4~TeV with 3000~fb$^{-1}$ of $pp$ collisions~\cite{CERN-LHCC-2017-018}. This mass limit is conservative due to the use of systematic uncertainties from the Run 1 analysis~\cite{Aad:2015fna}. 
These uncertainties are already smaller in the Run 2 analysis~\cite{Aaboud:2018juj} and will be further reduced at the time of the HL-LHC. In particular the systematic uncertainty in the boosted channel is now reduced due to the significant improvements of the performance of boosted jets in Run 2 (in particular using more tracking information to look for sub-jets within the large-radius jets). This gain in performance also improves the signal over background ratio. 

The usage of the top-tagger algorithm as employed in current Run-2 analyses will help to further reject background, hence the results presented should be considered conservative. 
